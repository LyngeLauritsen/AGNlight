%%%%%%%%%%%%%%%%%%%%%%%%%%%%%%%%%%%%%%%%%
% Journal Article
% LaTeX Template
% Version 1.3 (9/9/13)
%
% This template has been downloaded from:
% http://www.LaTeXTemplates.com
%
% Original author:
% Frits Wenneker (http://www.howtotex.com)
%
% License:
% CC BY-NC-SA 3.0 (http://creativecommons.org/licenses/by-nc-sa/3.0/)
%
%%%%%%%%%%%%%%%%%%%%%%%%%%%%%%%%%%%%%%%%%

%----------------------------------------------------------------------------------------
%	PACKAGES AND OTHER DOCUMENT CONFIGURATIONS
%----------------------------------------------------------------------------------------

\documentclass[a4paper]{article}

\usepackage{lipsum} % Package to generate dummy text throughout this template
\usepackage[utf8x]{inputenc}
\usepackage[T1]{fontenc}
\PrerenderUnicode{áéíóúñ}
%\usepackage[spanish]{babel}
%\usepackage{t1enc}
\usepackage[english]{babel}
\usepackage{graphicx}
\usepackage{sidecap}
\usepackage{amssymb,amsmath}
\usepackage{mathtools}
\usepackage{amsmath}
\usepackage{mathrsfs}
\usepackage{float} 
\usepackage[sc]{mathpazo} % Use the Palatino font
\usepackage[T1]{fontenc} % Use 8-bit encoding that has 256 glyphs
\linespread{1.05} % Line spacing - Palatino needs more space between lines
\usepackage{microtype} % Slightly tweak font spacing for aesthetics
\usepackage{url}
\usepackage[hmarginratio=1:1,top=15mm,columnsep=15pt,left=15mm]{geometry} % Document margins
\usepackage{multicol} % Used for the two-column layout of the document
\usepackage[hang, small,labelfont=bf,up,textfont=it,up]{caption} % Custom captions under/above floats in tables or figures
\usepackage{booktabs} % Horizontal rules in tables
\usepackage{float} % Required for tables and figures in the multi-column environment - they need to be placed in specific locations with the [H] (e.g. \begin{table}[H])
\usepackage{hyperref} % For hyperlinks in the PDF
\usepackage{caption}
\usepackage{lettrine} % The lettrine is the first enlarged letter at the beginning of the text
\usepackage{paralist} % Used for the compactitem environment which makes bullet points with less space between them
\def\textsubscript#1{\ensuremath{_{\mbox{\textscale{.6}{#1}}}}}
\usepackage{abstract} % Allows abstract customization
\renewcommand{\abstractnamefont}{\normalfont\bfseries} % Set the "Abstract" text to bold
\renewcommand{\abstracttextfont}{\normalfont\small\itshape} % Set the abstract itself to small italic text
\usepackage{titlesec} % Allows customization of titles
\graphicspath{ {pics/} }
\titleformat{\section}[block]{\large\scshape\centering}{\thesection.}{1em}{} % Change the look of the section titles
\titleformat{\subsection}[block]{\large}{\thesubsection.}{1em}{} % Change the look of the section titles
\renewcommand{\labelitemi}{$\bullet$}
\renewcommand{\labelitemii}{$\cdot$}
\renewcommand{\labelitemiii}{$\diamond$}
\renewcommand{\labelitemiv}{$\ast$}


%\usepackage{fancyhdr} % Headers and footers
%\pagestyle{fancy} % All pages have headers and footers
%\fancyhead{} % Blank out the default header
%\fancyfoot{} % Blank out the default footer
%\fancyhead[C]{Metallicity Gradients in ies $\bullet$ December 2015} % Custom header text
%\fancyfoot[RO,LE]{\thepage} % Custom footer text

%----------------------------------------------------------------------------------------
%	TITLE SECTION
%----------------------------------------------------------------------------------------

\title{\vspace{-20mm}\fontsize{16pt}{10pt}\selectfont\textbf{Continuum determination}} % Article title

\author{
%\large
\textsc{Lynge R. B. Lauritsen} \\
\normalsize University of Copenhagen \\ % Your institution
\date{March 2018}
\vspace{-9mm}
}
%----------------------------------------------------------------------------------------

\usepackage{amsmath}
\begin{document}
%\begin{multicols}{1}
\maketitle % Insert title
%\end{multicols}{}

%\thispagestyle{fancy} % All pages have headers and footers

%----------------------------------------------------------------------------------------
%	ABSTRACT
%----------------------------------------------------------------------------------------

%\begin{abstract}

%\noindent ABSTRACT

%\end{abstract}

%----------------------------------------------------------------------------------------
%	ARTICLE CONTENTS
%----------------------------------------------------------------------------------------
%\begin{multicols}{2} % Two-column layout throughout the main article text

\section{Introduction}
This paper will discuss the use of Reverberation Mapping of AGN Light Curves on quasars observed using the Rapid Eye Mount (REM) telescope in the La Palma site in Chile. The aim of this paper is to demonstrate the possibilities in using observation from different bands to determine the driving function of the quasars' as well as all relevant transfer functions. It will discuss the process of determining the light curves based on local standard stars, as well as the creation and running of the MCMC algorithm used to determine the relevant constants of the transfer functions.

\section{Active Galactic Nuclei}
Active Galactic Nuclei, or AGN for short, is used to describe powerful energetic and luminous phenomena in the galactic center that does not originate from the galactic stellar population. These phenomena is powered by accretion onto a Supermassive Black Hole (SMBH). This section will discuss the AGN phenomena, how it originates as well as its importance in modern astrophyics.

\subsection{Observing and identifying the AGN}
AGN identification and observation follow a list of observable properties. The earlier identification methods as used by both Schmidt (1969) as well as Peterson (1997) allows AGN identification based upon datasets of similar nature as the ones used in this project. These identification properties is; 
\begin{enumerate}
\item The pointlike representation of an AGN upon an imaging detector
\item Strong emission lines
\item The Continuum Luminosity varies over time
\item Evidence of strong non-stellar emission
\end{enumerate}
it must however be added that the Peterson (2008) expanded upon this list
\begin{enumerate}
\item Strong X-RAY emission
\item Radio emission
\item Non-stellar UV through IR emission
\item Broad emission lines in the UV through IR.
\end{enumerate}
of the latter list however it is important to note that not necessarily all items is always registered. \\
\\
\subsection{AGN morphology}
It can generally be said that AGN falls into two different categories. The Seyfert Galaxies and the Quasars. These categories each have identifying points, however it is somewhat unclear to which degree they are distinctly different objects, or if the difference is mostly due to possibilities of observation. The distinction is of general importance and interest, however for the majority of this project it makes little actual difference in the results obtained, although it should be said that the available data did belong to Seyfert galaxies. 

\subsubsection{Seyfert Galaxies}
Discovered by Carl Seyfert (1943) these galaxies are Active Galaxies characterized by being spiral galaxies with a bright star-like nuclei in the center. Spectroscopically these galaxies contains both non-thermal continuum radiation and broad emission lines in their spectra. In addition observations show that the Luminosity output from the Nuclei originating in these Active Galaxies can at times vary by more than a factor of two in a year. \\
\\
The first Active Galaxy observed in 1908 by E. A. Fath at the Lick Observatory was NGC 1068. However not until 1943 did Carl Seyfert identify these as a distinct class of galaxies, now called Seyfert Galaxies. Not until the late 1950's did these galaxies become relevant again, with their identification as radio sources. Woltjer (1959) identified these Seyfert galaxies as having:
\begin{enumerate}
\item Unresolved nuclei, so at the then observational quality, a nucleus smaller than 100 pc. 
\item Lifetimes in excess of $10^8$ years. This being concluded from the realisation that Seyfert Galaxies makes up 1/100 of all spiral galaxies. Leading to two possible conclusions. Either all Spiral Galaxies pases through a Seyfert phase, or they are fundamentally different from other spirals, making it logical to assume they have lifetimes comparable to other spiral galaxies (of order $10^{10}$ yrs).
\item If it is assumed the material in the Nucleus is gravitationally bound, then based on the widths of the emission lines (excess of $10^3$ $kms^{-1}$) and the viral argument \emph{equation \ref{eq:Viral}},

\begin{equation}
M \approx \frac{v^2r} {G}
\label{eq:Viral}
\end{equation}
then it must be assumed the Mass of the Nucleus is of the order $10^6$ $M_O$.
\end{enumerate}
It is of note that the Seyfert Galaxies fit into two types. Seyfert I galaxies has permitted emission lines originating primarily from Hydrogen with very broad characteristics and giving FWHM corresponding to velocities in excess of $10^3$ $kms^{-1}$. Additionally the Seyfert I galaxies also contain forbidden lines (such as [OIII]) with much narrower profiles ($10^2-10^3$ $kms^{-1}$). Seyfert II galaxies differs in that the emission observed is originating in the Narrow Line Region (NLR) entirely. This does not necessarily imply an absence of a Broad Band Region (BLR), as this region could be unobservable in this galaxy.

\subsubsection{Quasars}
Quasars, or as originally called Quasi-Stellar Radio Source, are the most luminous AGNs observed. A subset of these are also strong radio sources (5-10\%), and it was originally these objects that defined the quasars distinction and name. Generally quasars have strong similarities with Seyfert Galaxies, however they have very weak stellar absorption features and the Narrow Lines tends to be weaker when compared to the Broad Lines than observed in the Seyfert Galaxies. The optical spectrum observed from quasars are similar to those observed in the Seyfert Galaxies. In formal classification the distinction is made from the absolute magnitude with \emph{equation \ref{eq:quasar_distinction}} defining a quasar.
\begin{equation}
M_{B} \le -21.5 + 5log(h_{0}) 
\label{eq:quasar_distinction}
\end{equation}
This distinction is however a historical construct, based upon the observable qualities of the respective AGNs, as it appears the only fundamental difference between Quasars and Seyfert Galaxies is the Luminosity of the object. The quasar Luminosity can be of the order $10^3$ larger than that of a galaxy, and therefore only at low redshift high resolution quasars will the host galaxy be observable.[CHAPTER 14 GALAXY FORMATION AND EVOLUTION]

\subsubsection{Radio Galaxies}
The normal spiral galaxies will have weak radio emission (mostly powered through SN remnants) and therefore have power (\emph{equation \ref{eq:SN_radio}}),

\begin{equation}
P_{1.4GHz} \le 2*10^{23} WHz^{-1}
\label{eq:SN_radio}
\end{equation}
this allows the definition of radio galaxies of being galaxies with $P_{1.4GHz}$ larger than \emph{equation \ref{eq:SN_radio}}. \\
\\
Radio Galaxies were originally identified through the third survey at Cambridge, and it has since been realised that almost all radio galaxies are AGN ellipticals. Much like the Seyfert Galaxies two types are identified, the Broad-Line Radio Galaxies (BLRG) and Narrow-Line Radio Galaxies (NLRG). The BLRG and NLRG differs from their corresponding Seyfert Galaxies partly in being radio loud, and the morphology of the host galaxy, but also in the existence of mostly asymmetric stretching radio jets stretching several hundred kiloparsec or even megaparsec from the AGN. \\

\subsection{AGN structure}
The AGN is situated around a SMBH, and is powered by gas being accreted onto the central SMBH. The energy originates from the potential energy stored in the accretion disk with respect to the SMBH. 


\section{Light Curve Creation}



\section{Finding Continuum}
In this section the process utilised in determining the used continuum destribution of the observed quasar data is described. The method used was found through a combination of reading relevant literature and implementation of numerical MCMC algorithm. At no point in this endeavor has the aim been to find the absolute Continuum Light Curve (CLC). The objectively correct CLC is of no real importance in the investigation, and therefore would cause unnecessary time to pursue. The difficulty in determining the absolute CLC is in the lack of knowledge of the actual band dependent transfer function of the observed Light Curves (LC). Additionally the interest in the project is the timelag between the observable bands, and as such the relative transfer functions as opposed the the absolute transfer functions. \\
This section will be focused upon describing the methods used and the reasons behind the decisions taken.\\

\subsection{Original Data Material and Kelly manipulations}
The CLC is found based upon the observed light curves of the K-band. The REM data is observed in the KHJgriz bands. The REM data is uneven in the sampling and subject to several observation gaps of a 50 - 100 day period. Due to this sampling it has proved of interest to attempt to simulate the Observed Light Curves (OLC) across the observational gaps. These has been filled through the use of the Kelly Function (Kelly et al. 2009). The Kelly Function is not actually a function as much as a way of approximating the next point of the LC based upon the overall distribution of observed values. It is given by \emph{equation \ref{eq:Kelly2009}},

\begin{equation}
dX(t) = -\frac{1}{\tau}X(t)dt + \sigma\sqrt{dt}\epsilon(t) + bdt
\label{eq:Kelly2009}
\end{equation}
with \emph{b} being the observed mean value of the OLC and $\tau$ is the relaxation time. The Kelly approach introduces a bias designed to pull the LC towards the mean. To counter this offset the Kelly approach has been applied in both directions of the LC and the mean of the two functions is the accepted value. The $\epsilon$ is a white noise process with mean zero and standard deviation of one. This is examplified in \emph{figure \ref{fig:NGC3783K-Kelly}}

\begin{figure}[htp!]
\centering
\includegraphics[width=1\linewidth]{/home/lynge/MasterP/Figure/NGC3783K-Kelly.png}\\
\caption{The Kelly function applied to the NGC3783 K-band spectrum.}
\label{fig:NGC3783K-Kelly}
\end{figure}

The weakness of the Kelly method becomes clear in the second interval of missing data points (around 57800 - 57900 MJD). The Kelly Function breaks down in this area, it may be possible to adjust this somewhat by introducing a dependence of time from the points of observation that the estimate is based upon. This however is not the focus at this time. \\
\\
\subsection{Transfer Functions}
In order to determine the CLC one must have an understanding of how the LC behaves from the Quasar to the observation. If one were to determine the exact Transfer Function at all times, it would then be possible to determine the exact CLC. However the transfer function is an unknown quantity and as the OLC is the result of the transfer function and the OLC (\emph{equation \ref{eq:OLC}}) (Andreas Skielboe 2016)
\begin{equation}
F_l(t,\lambda) = \int_{-\infty}^{\infty}\Psi(\tau,\lambda)F_C(t-\tau)d\tau
\label{eq:OLC}
\end{equation}
it is impossible to accurately determine the CLC. However this project is not concerned with the accurate CLC, it is however interested in the relative difference between the Transfer Functions. It is therefore decided to assume a Transfer Function for the K-band data. Using this arbitrary function, \emph{equation \ref{eq:OLC}} and an MCMC algorithm a possible CLC is determined. This possible CLC can then be utilised in compound with the OLC for the renmaining observed bands and \emph{equation \ref{eq:OLC}} to determine the relative differences and hence the timelag between the Transfer Functions. \\
For arbitrary Transfer Function a log-normal is chosen (\emph{equation \ref{eq:TF}})
\begin{equation}
f(x) = \frac{1}{x\sigma\sqrt{2\pi}}e^{-\frac{(ln(x))^2}{2\sigma^2}}
\label{eq:TF}
\end{equation}

\subsection{Power Spectral Density}
OLC from AGN's has distinct Power Spectral Densities (PSD's). This detail is used to determine whether the outcome from the MCMC algorithm is in fact a CLC, or just one of infinitely many possible solutions that exists to the numerical solving of \emph{equation \ref{eq:TF}}. The PSD slope is generally in the vicinity -2 to -3. The PSD is given by \emph{equation \ref{eq:PSD}} (Uttley et al. 2002).
\begin{equation}
P(\nu) = \frac{2T}{\mu^2N^2}|F_N(\nu)|^2
\label{eq:PSD}
\end{equation}
with $|F_N(\nu)|^2$ given by \emph{equation \ref{eq:F_N}}
\begin{equation}
|F_N(\nu)|^2 = [\sum_{i=1}^{N}f(t_i)cos(2\pi\nu t_i)]^2 + [\sum_{i=1}^{N}f(t_i)sin(2\pi\nu t_i)]^2
\label{eq:F_N}
\end{equation}

\subsection{MCMC algorithm and reasoning}
The CLC is determined through an MCMC type algorithm. An initial guess for the CLC is made and the quality of the fit is made through the use of a variety of factors. 
\begin{enumerate}
\item Determining the residuals squared of the $F_l(t,\lambda)$
\item Determining the double derivative of the CLC
\item Determining the PSD slope of the CLC
\item Producing a Kelly fitting for the CLC
\end{enumerate}
The code then randomly alters the first point on the CLC and item 1 through 4 is redetermined and compared. In the case of a favorable outcome the alteration is saved and the code moves onwards to the following point. The favorability of an outcome is evalueated by a series of parameters. 
\begin{enumerate}
\item Residuals: In all cases the sum of the residuals squared must be less than the previous alteration.
\item Double Derivative: The double derivative is compared to the maximum rate of change of the OLC and is accepted if it is no more than 40 percent larger than the originally observed. This is done to prevent rapid changes to the CLC that would ultimately make for a more stable, but ultimately unphysical solution to the CLC. 40 percent has been chosen as it is felt that despite the OLC becoming somewhat more smooth as a result of the Transfer Function, it would be unlikely to be that prominent. The alternative is the sum of the change in the rate of change of beth adjacent points as well as the altered points decreases overall. This would be accepted as well, pending other factors.
\item PSD slopes: Assuming 1 and 2 holds true, the change can be accepted if the PSD slope is moving closer to the accepted slope, or inside 0.05 of the accepted (so as to allow some freedom of movement of the CLC).
\item Kelly: In the case of 1 holding true, and 2 follows the path of the set of double derivatives overall decreasing there will be a statistical possibility of 5 percent of a change being accepted IF the Kelly function provides an overall better fit and the PSD slope is no more than 0.3 out. This is done primarily to utilise the Kelly function as a method of approximating LC's and hence allowing for the use of this additional resource in providing a more physical fitting, as well as counterbalancing the possibility of the CLC becoming stable in an unstable equilibrium position due to the other limitations.
\end{enumerate}
It is being experimented upon with both one moving point as well as three. 




%----------------------------------------------------------------------------------------
%	REFERENCE LIST
%----------------------------------------------------------------------------------------
\newpage
[1] Kelly et al. 2009, APJ698:895-910 
[2] Kelly et al. 2009, arXiv:0903.5315v1 
[3] A. Skielboe 2016, Thesis 
[4] Uttley et al. 2002 Mon. Not. R. Astron. Soc. 332,231-250

%\end{multicols}
\end{document}
